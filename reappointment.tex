% This template was adapted from https://github.com/tpavlic/latex_cv_template

%%%%%%%%%%%%%%%%%%%%%%%%%%%%%%%%%%%%%%%%%%%%%%%%%%%%%%%%%%%%%%%%%%%%%%%%
%%%%%%%%%%%%%%%%%%%%%% Simple LaTeX CV Template %%%%%%%%%%%%%%%%%%%%%%%%
%%%%%%%%%%%%%%%%%%%%%%%%%%%%%%%%%%%%%%%%%%%%%%%%%%%%%%%%%%%%%%%%%%%%%%%%

%%%%%%%%%%%%%%%%%%%%%%%%%%%%%%%%%%%%%%%%%%%%%%%%%%%%%%%%%%%%%%%%%%%%%%%%
%% NOTE: If you find that it says                                     %%
%%                                                                    %%
%%                           1 of ??                                  %%
%%                                                                    %%
%% at the bottom of your first page, this means that the AUX file     %%
%% was not available when you ran LaTeX on this source. Simply RERUN  %%
%% LaTeX to get the ``??'' replaced with the number of the last page  %%
%% of the document. The AUX file will be generated on the first run   %%
%% of LaTeX and used on the second run to fill in all of the          %%
%% references.                                                        %%
%%%%%%%%%%%%%%%%%%%%%%%%%%%%%%%%%%%%%%%%%%%%%%%%%%%%%%%%%%%%%%%%%%%%%%%%

%%%%%%%%%%%%%%%%%%%%%%%%%%%% Document Setup %%%%%%%%%%%%%%%%%%%%%%%%%%%%

% Don't like 10pt? Try 11pt or 12pt
\documentclass[11pt]{report}
\usepackage[text]{datetime}
\usepackage{times}
% Finally, give us PDF bookmarks
\usepackage{color}
\definecolor{darkblue}{rgb}{0.0,0.0,0.7}
\usepackage{hyperref}
\hypersetup{colorlinks,breaklinks,
            linkcolor=darkblue,urlcolor=darkblue,
            anchorcolor=darkblue,citecolor=darkblue}
%\newcommand{\href}[2]{#2}

% This is a helpful package that puts math inside length specifications
\usepackage{calc}
%\usepackage{acl-partial}
%\usepackage{acl-hlt2011}
% MRG: DISABLE NATBIB FOR BIBLATEX: 
%\usepackage{natbib}
\usepackage{ amssymb }
\usepackage[warn]{ textcomp }
\usepackage{comment}
\usepackage{graphicx}
\usepackage{float}
\usepackage{subcaption}
\usepackage{booktabs}
\usepackage{paralist}
\usepackage{tabto}
\usepackage{ifthen}
\usepackage{environ}
% For lorem ipsum text
\usepackage{lipsum}

% Use this for better spacing of subsections and smallcaps
% section headers:
\usepackage[uppercase]{titlesec}
\titlespacing*{\chapter}{0pt}{0pt}{10pt}
% \titleformat{\chapter}[display]
% {\sc\Large\filcenter}
% {\LARGE\MakeUppercase{\chaptertitlename} \thechapter}
% {1pc}
% {\Huge}
\titleformat{\chapter}[display]
     {\centering\bfseries\Large}
     {\Huge\thechapter}
     {0pt}
     {\MakeUppercase}
     []
\titleformat{\paragraph}[runin]{\normalfont\normalsize\bfseries}{\theparagraph}{1em}{}

% % Simpler bibsection for CV sections
% % (thanks to natbib for inspiration)
% \makeatletter
% \newlength{\bibhang}
% \setlength{\bibhang}{1em}
% \newlength{\bibsep}
%  {\@listi \global\bibsep\itemsep \global\advance\bibsep by\parsep}
% \newenvironment{bibsection}%
%         {\begin{list}{}{%
%        \setlength{\leftmargin}{\bibhang}%
%        \setlength{\itemindent}{-\leftmargin}%
%        \setlength{\itemsep}{\bibsep}%
%        \setlength{\parsep}{\z@}%
%         \setlength{\partopsep}{0pt}%
%         \setlength{\topsep}{0pt}}}
%         {\end{list}\vspace{-.6\baselineskip}}
% \makeatother

% Layout: Puts the section titles on left side of page
%\reversemarginpar

%
%         PAPER SIZE, PAGE NUMBER, AND DOCUMENT LAYOUT NOTES:
%
% The next \usepackage line changes the layout for CV style section
% headings as marginal notes. It also sets up the paper size as either
% letter or A4. By default, letter was used. If A4 paper is desired,
% comment out the letterpaper lines and uncomment the a4paper lines.
%
% As you can see, the margin widths and section title widths can be
% easily adjusted.
%
% ALSO: Notice that the includefoot option can be commented OUT in order
% to put the PAGE NUMBER *IN* the bottom margin. This will make the
% effective text area larger.
%
% IF YOU WISH TO REMOVE THE ``of LASTPAGE'' next to each page number,
% see the note about the +LP and -LP lines below. Comment out the +LP
% and uncomment the -LP.
%
% IF YOU WISH TO REMOVE PAGE NUMBERS, be sure that the includefoot line
% is uncommented and ALSO uncomment the \pagestyle{empty} a few lines
% below.
%

%% Use these lines for letter-sized paper
\usepackage[paper=letterpaper,
            %includefoot, % Uncomment to put page number above margin
            %marginparwidth=1.2in,     % Length of section titles
            %marginparsep=.05in,       % Space between titles and text
            margin=1in,               % 1 inch margins
            ]{geometry}

%% More layout: Get rid of indenting throughout entire document
%\setlength{\parindent}{0in}

\usepackage[shortlabels]{enumitem}

%% Reference the last page in the page number
%
% NOTE: comment the +LP line and uncomment the -LP line to have page
%       numbers without the ``of ##'' last page reference)
%
% NOTE: uncomment the \pagestyle{empty} line to get rid of all page
%       numbers (make sure includefoot is commented out above)
%
\usepackage{fancyhdr,lastpage}
% \pagestyle{fancy}
\pagestyle{empty}      % Uncomment this to get rid of page numbers
\fancyhf{}\renewcommand{\headrulewidth}{0pt}
\fancyfootoffset{\marginparsep+\marginparwidth}
\newlength{\footpageshift}
\setlength{\footpageshift}
          {0.5\textwidth+0.5\marginparsep+0.5\marginparwidth-2in}
\lfoot{\hspace{\footpageshift}%
       \parbox{4in}{\, \hfill %
                    \arabic{page} of \protect\pageref*{LastPage} % +LP
%                    \arabic{page}                               % -LP
                    \hfill \,}}

% ===================
% BibLaTeX
% =================== 
\usepackage[style=numeric, backend=bibtex,
sorting=ydnt,
defernumbers=true,
maxbibnames=99, 
url=false, doi=false, isbn=false,
hyperref=true,
maxcitenames=2]{biblatex}

% Never show the month.
\AtEveryBibitem{
  \clearfield{month}{}
  \clearlist{language}{}
  \clearfield{editor}{}
  \clearfield{pages}
}
% Add URL to title.
\DeclareNameAlias{sortname}{first-last}
\newbibmacro{string+doiurlisbn}[1]{%
  \iffieldundef{doi}{%
    \iffieldundef{url}{%
      \iffieldundef{isbn}{%
        \iffieldundef{issn}{%
          #1%
        }{%
          \href{http://books.google.com/books?vid=ISSN\thefield{issn}}{#1}%
        }%
      }{%
        \href{http://books.google.com/books?vid=ISBN\thefield{isbn}}{#1}%
      }%
    }{%
      \href{\thefield{url}}{#1}%
    }%
  }{%
    \href{http://dx.doi.org/\thefield{doi}}{#1}%
  }%
}
\DeclareFieldFormat{title}{\usebibmacro{string+doiurlisbn}{\mkbibemph{#1}}}
\DeclareFieldFormat[article,inproceedings]{title}%
{\usebibmacro{string+doiurlisbn}{\mkbibquote{#1}}}
\addbibresource{mypapers.bib}
\addbibresource{mysoftware.bib}
 
\AtEveryBibitem{
  \clearfield{address}
  \clearfield{location}
}

%%%%%%%%%%%%%%%%%%%%%%%% End Document Setup %%%%%%%%%%%%%%%%%%%%%%%%%%%%


%%%%%%%%%%%%%%%%%%%%%%%%%%% Helper Commands %%%%%%%%%%%%%%%%%%%%%%%%%%%%

% The title (name) with a horizontal rule under it
% (optional argument typesets an object right-justified across from name
%  as well)
%
% Usage: \makeheading{name}
%        OR
%        \makeheading[right_object]{name}
%
% Place at top of document. It should be the first thing.
% If ``right_object'' is provided in the square-braced optional
% argument, it will be right justified on the same line as ``name'' at
% the top of the CV. For example:
%
%       \makeheading[\emph{Curriculum vitae}]{Your Name}
%
% will put an emphasized ``Curriculum vitae'' at the top of the document
% as a title. Likewise, a picture could be included:
%
%   \makeheading[\includegraphics[height=1.5in]{my_picutre}]{Your Name}
%
% the picture will be flush right across from the name.
\newcommand{\makeheading}[2][]%
        {\hspace*{-\marginparsep minus \marginparwidth}%
         \begin{minipage}[t]{\textwidth+\marginparwidth+\marginparsep}%
             {\large \bfseries #2 \hfill #1}\\[-0.15\baselineskip]%
                 \rule{\columnwidth}{1pt}%
         \end{minipage}}

% The section headings
%
% Usage: \section{section name}
% \renewcommand{\section}[1]{\pagebreak[3]%
%     \hyphenpenalty=10000%
%     \vspace{1.3\baselineskip}%
%     %\phantomsection\addcontentsline{toc}{section}{#1}%
%     \noindent\llap{\scshape\smash{\parbox[t]{\marginparwidth}{\raggedright #1}}}%
%     \vspace{-\baselineskip}\par}
%
% \renewcommand{\subsection}[1]{{\addvspace{1em}\noindent\small\textsc{#1}}\vspace{.5em}}
  
% An itemize-style list with lots of space between items
% \newenvironment{outerlist}[1][\enskip\textbullet]%
%         {\begin{itemize}[#1,leftmargin=*]}{\end{itemize}%
%          \vspace{-.6\baselineskip}}

\newenvironment{outerlist}[1][\enskip\textbullet]%
        {\begin{list}{-}{
              \setlength{\itemsep}{0pt}
              \setlength{\parsep}{1pt}
              \setlength{\topsep}{2pt}
              %\setlength{\labelwidth}{10em}
              \setlength{\leftmargin}{2em}
             }}
          {\end{list}}

\newenvironment{innerlist}[1][\enskip\textbullet]%
        {\begin{list}{\raisebox{.1em}{{\small \textopenbullet}}}{
              \setlength{\itemsep}{0pt}
              \setlength{\parsep}{1pt}
              \setlength{\topsep}{0pt}
              %\setlength{\labelwidth}{10em}
              \setlength{\leftmargin}{2em}
             }}
          {\end{list}}

\newenvironment{lonelist}[1][\enskip\textbullet]%
        {\begin{list}{-}{
              \setlength{\topsep}{0pt}
              \setlength{\itemsep}{2pt}
              \setlength{\parsep}{1pt}
              %\setlength{\labelwidth}{1em}
              %\setlength{\labelsep}{1em}
              \setlength{\leftmargin}{1em}
             }}
          {\end{list}}

\newenvironment{biblist}
        {\begin{list}{}{
        \setlength{\topsep}{0pt}
       \setlength{\itemsep}{2pt}
       \setlength{\leftmargin}{1em}
       \setlength{\itemindent}{-\leftmargin}
       \setlength{\parsep}{1pt}
        \setlength{\partopsep}{0pt}}}
       {\end{list}}

\newcommand{\verbose}[1]{}

% % An environment IDENTICAL to outerlist that has better pre-list spacing
% % when used as the first thing in a \section
% \newenvironment{lonelist}[1][\enskip\textbullet]%
%         {\begin{list}{#1}{%
%         \setlength{\partopsep}{0pt}%
%         \setlength{\topsep}{0pt}}}
%         {\end{list}\vspace{-.6\baselineskip}}

% % An itemize-style list with little space between items
% \newenvironment{innerlist}[1][\enskip\textbullet]%
%         {\begin{itemize}[#1,leftmargin=*,parsep=0pt,itemsep=0pt,topsep=0pt,partopsep=0pt]}
%         {\end{itemize}}

% An environment IDENTICAL to innerlist that has better pre-list spacing
% when used as the first thing in a \section
\newenvironment{loneinnerlist}[1][\enskip\textbullet]%
        {\begin{itemize}[#1,leftmargin=*,parsep=0pt,itemsep=0pt,topsep=0pt,partopsep=0pt]}
        {\end{itemize}\vspace{-.6\baselineskip}}

% To add some paragraph space between lines.
% This also tells LaTeX to preferably break a page on one of these gaps
% if there is a needed pagebreak nearby.
%\newcommand{\blankline}{\quad\pagebreak[3]}
%\newcommand{\halfblankline}{\quad\vspace{-0.5\baselineskip}\pagebreak[3]}
%\setlength{\parskip}{0.5em}

% For \url{SOME_URL}, links SOME_URL to the url SOME_URL
\providecommand*\url[1]{\href{#1}{#1}}
% Same as above, but pretty-prints SOME_URL in teletype fixed-width font
\renewcommand*\url[1]{\href{#1}{\texttt{#1}}}

% For \email{ADDRESS}, links ADDRESS to the url mailto:ADDRESS
\providecommand*\email[1]{\href{mailto:#1}{#1}}
% Same as above, but pretty-prints ADDRESS in teletype fixed-width font
%\renewcommand*\email[1]{\href{mailto:#1}{\texttt{#1}}}

%\providecommand\BibTeX{{\rm B\kern-.05em{\sc i\kern-.025em b}\kern-.08em
%    T\kern-.1667em\lower.7ex\hbox{E}\kern-.125emX}}
%\providecommand\BibTeX{{\rm B\kern-.05em{\sc i\kern-.025em b}\kern-.08em
%    \TeX}}
\providecommand\BibTeX{{B\kern-.05em{\sc i\kern-.025em b}\kern-.08em
    \TeX}}
\providecommand\Matlab{\textsc{Matlab}}

\newcommand{\dates}[1]{\hfill \textit{#1}}
\newcommand{\location}[1]{\hfill #1}
\newcommand{\position}[1]{\\\textit{#1}}
\newcommand{\project}[1]{\textbf{#1}}
\newcommand{\institution}[1]{\textit{#1}}

% Highlights
%\newcommand{\hlight}[1]{\textcolor{blue}{#1}}
\newcommand{\hlight}[1]{#1}
%\newcommand{\bstart}[1]{\textbf{#1}}
\newcommand{\bstart}[1]{#1}

% Do not show section numbering
\setcounter{secnumdepth}{-1}


% REDACTED
% Command for eliding other instructors
\newcommand{\oinstr}[1]{SCS professor}
\newcommand{\oinstrs}[1]{SCS professors (co-taught)}

% To change between a CMU Internal version (set to value 1) of this CV for reappointment
% and an external one (set to value 0).
\newcommand{\cmuinternal}{1}

\ifthenelse{\equal{\cmuinternal}{1}}{
  \NewEnviron{internalonly}{\BODY} % Visible
  \NewEnviron{externalonly}{} % Invisible
}{
  \NewEnviron{internalonly}{} % Invisible
  \NewEnviron{externalonly}{\BODY} % Visible
}



%%%%%%%%%%%%%%%%%%%%%%%% End Helper Commands %%%%%%%%%%%%%%%%%%%%%%%%%%%

%%%%%%%%%%%%%%%%%%%%%%%%% Begin CV Document %%%%%%%%%%%%%%%%%%%%%%%%%%%%

\begin{document}

\begin{internalonly}
\tableofcontents
\end{internalonly}

\chapter{Curriculum Vitae}

\begin{externalonly}

\section{Contact}

% NOTE: Mind where the & separators and \\ breaks are in the following
%       table.
%
% ALSO: \rcollength is the width of the right column of the table
%       (adjust it to your liking; default is 1.85in).
%
\newlength{\rcollength}\setlength{\rcollength}{2.5in}%
%
\begin{tabular}[t]{@{}p{\linewidth-\rcollength}p{\rcollength}}
{\large \bf Grace Hopper} & \\
Yale University   & \textit{E-mail:} \email{ghopper@yale.edu}\\
38 Hillhouse Avenue  & \textit{Phone:} 412-779-2018 \\
New Haven, CT 06520    &       \href{http://www.yale.edu/~ghopper/}{http://www.yale.edu/~ghopper/}
\end{tabular}
\end{externalonly}

\textit{IMPORTANT NOTE: The contents of this template are entirely generated by a large language model and are solely intended as an example for how to format this document. The contents were not checked for historical accuracy.}

\section{Education}

\begin{itemize}

\item \textbf{Yale University} \dates{1930--1934}\\
\textit{Graduate School of Arts and Sciences}
\begin{outerlist}
\item Ph.D. in Mathematics, {1934}
  \begin{innerlist}
  \item Dissertation: “New Types of Irreducibility Criteria”
  \item One of the first women to earn a Ph.D. in Mathematics from Yale
  \end{innerlist}
\item M.A. in Mathematics, {1930}
\end{outerlist}

\item \textbf{Vassar College} \dates{1924--1928}\\
\textit{Department of Mathematics and Physics}
\begin{outerlist}
\item B.A. in Mathematics and Physics, {1928}
  \begin{innerlist}
  \item Graduated Phi Beta Kappa
  \item Valedictorian-level academic standing
  \end{innerlist}
\end{outerlist}

\end{itemize}

\section{Employment}

\begin{itemize}

\item \textbf{United States Navy} \dates{1943--1986} 
\position{Rear Admiral (lower half)}, United States Naval Reserve 
\position{Pioneer in Automatic Programming and Compiler Design}
\vspace{1em}

\item \textbf{Remington Rand / Sperry Corporation} \dates{1949--1966} 
\position{Senior Mathematician}, UNIVAC Division 
\position{Lead Developer}, A-0 Compiler and Early COBOL Standardization
\vspace{1em}

\item \textbf{Harvard University} \dates{1944--1949} 
\position{Research Fellow}, Computation Laboratory 
\position{Programmer}, Mark I and Mark II Computers under Howard Aiken
\vspace{1em}

\item \textbf{Vassar College} \dates{1931--1943} 
\position{Associate Professor}, Department of Mathematics
\vspace{1em}

\end{itemize}

%SKIP: \section{Personal}

\begin{internalonly}
  
\chapter{Statement of Career Goals}

% Fill with lorem ipsum text
\lipsum[1-2]


\end{internalonly}


\chapter{Publications List}

% NEW WAY
% Using biblatex package to create multiple bibliographies,
% subdividing by type or keyword.
%
% http://texblog.org/2012/10/22/multiple-bibliographies-with-biblatex/
% http://tex.stackexchange.com/questions/48023/mimic-bibtex-apalike-with-biblatex-biblatex-apa-broken
%

\nocite{*}
\setlength\bibitemsep{7pt}
\printbibliography[heading=subbibliography,title={Refereed Journal
  Papers - Published},type=article]
\printbibliography[heading=subbibliography,title={Refereed Conference/Workshop
  Papers},type=inproceedings]
\printbibliography[heading=subbibliography,title={Technical
  Reports},type=report]
\printbibliography[heading=subbibliography,title={Other
  Publications},type=thesis]
\printbibliography[heading=subbibliography,title={Software Artifacts},type=misc]


\begin{externalonly}
\chapter{Funding}
\begin{lonelist}
\item Generous Donor [\$1] \dates{2099}
\item Really Generous Donor [\$2] \dates{2098}
\end{lonelist}
\end{externalonly}

\begin{internalonly}
\chapter{Recent Publications}

Numeric citations below refer to the {\sc publications list} above.
\begin{itemize}
  \item \cite{hopper_compiling_1988} Hopper (1988)
  \item \cite{hopper_automatic_1955} Hopper (1955)
  \item \cite{hopper_education_1952} Hopper (1952)
\end{itemize}
  
\chapter{Contract and Grant Support}
Supplied by Department.

\chapter{Faculty Course Evaluations}
Supplied by Department.

\chapter{Statement of Teaching Philosophy and Self Evaluation}

\lipsum[1-2]

\end{internalonly}

\chapter{Contributions to Education}

\begin{internalonly}

\textit{Below are some ideas for sections to include in this chapter.}

\section{Curriculum Design}

\begin{itemize}
\item \lipsum[1]
\item \lipsum[2]
\item \lipsum[3]
\end{itemize}

\section{Teaching Materials}

\begin{itemize}
\item \lipsum[1]
\item \lipsum[2]
\item \lipsum[3]
\end{itemize}

\section{Assessments}

\begin{itemize}
\item \lipsum[1]
\item \lipsum[2]
\item \lipsum[3]
\end{itemize}


\section{New Programs}

\begin{itemize}
\item \lipsum[1]
\item \lipsum[2]
\item \lipsum[3]
\end{itemize}

\end{internalonly}


%\chapter{Evidence of Teaching Performance}

\newcommand{\crsNote}[1]{\tabto{6.1cm}[#1]}

\section{Courses Taught At Carnegie Mellon}

\textit{Below is an example of how to possibly format these lists.}
\begin{itemize}

\item
  \href{https://www.vassar.edu/}
  {Mathematics 101: Introduction to Mathematical Reasoning}
  \dates{Fall 1935}\\
  \emph{Foundations of logic, set theory, and proof techniques for undergraduates}\\
  35 students

\item
  \href{https://www.vassar.edu/}
  {Mathematics 205: Differential and Integral Calculus}
  \dates{Spring 1938}\\
  \emph{Classical calculus with applications to physics and engineering}\\
  42 students

\item
  \href{https://www.vassar.edu/}
  {Mathematics 312: Applied Mathematics and Numerical Methods}
  \dates{Spring 1942}\\
  \emph{Focus on mathematical techniques relevant to wartime computation}\\
  27 students

\end{itemize}

\section{Courses Taught Outside Carnegie Mellon}

\begin{itemize}

\item
  \href{https://www.vassar.edu/}
  {Mathematics 398: Seminar on Logic and Machines}
  \dates{Fall 1943}\\
  \emph{Independent study course introducing students to formal logic and mechanical computation concepts}\\
  12 students

\end{itemize}

\section{Other Courses Taught}

\begin{itemize}

\item Summer School course

\end{itemize}

\begin{internalonly}
\chapter{Professional Activities}
\end{internalonly}

\section{Conference and Workshop Committees}

\begin{itemize}
\item
  National Meeting on Automatic Programming, U.S. Navy \dates{1952} \\
  \emph{Served as organizer and lead speaker on automatic compilers}
\item
  ACM National Conference \dates{1976} \\
  \emph{Keynote Speaker: "The Future of Computers"}
\end{itemize}

\section{Consulting}

\begin{itemize}
\item
  \textbf{Remington Rand / Sperry Univac} \dates{1949--1966} \\
  Consultant and senior mathematician for UNIVAC software systems
\item
  \textbf{U.S. Department of Defense} \dates{1967--1986} \\
  Advised on programming language standardization (especially COBOL)
\end{itemize}

\section{Memberships in Professional Societies}

\begin{itemize}
\item Fellow, Association for Computing Machinery (ACM)
\item Fellow, Institute of Electrical and Electronics Engineers (IEEE)
\item Member, American Association for the Advancement of Science (AAAS)
\item Member, U.S. Navy League
\end{itemize}

\section{Editorial Board Memberships}

\begin{itemize}
\item Advisory contributor to technical documentation for the COBOL standard \dates{1960--1963}
\end{itemize}

\section{Other}

\begin{itemize}
\item Namesake of the annual \href{https://ghc.anitab.org/}{Grace Hopper Celebration of Women in Computing} \\
  \emph{The world's largest gathering of women in technology}
\item Recipient of the National Medal of Technology (1991, posthumously)
\item Namesake of the USS Hopper (DDG-70), a U.S. Navy destroyer
\end{itemize}


\begin{internalonly}
\chapter{Evidence of External Reputation}
\end{internalonly}
\section{Citations and Awards}

\begin{itemize}
\item National Medal of Technology, \institution{President George H. W. Bush, United States} \dates{1991 (posthumously)}
\item IEEE Emanuel R. Piore Award \institution{IEEE} \dates{1988}
\item ACM Distinguished Service Award \institution{Association for Computing Machinery} \dates{1971}
\item Navy Meritorious Service Medal \institution{United States Navy} \dates{1980}
\item First Computer Science Man-of-the-Year Award \institution{Data Processing Management Association} \dates{1969}
\end{itemize}

\section{Invited Talks}
\begin{itemize}
\item Keynote Address, \institution{ACM National Conference} \dates{1976} \\
\emph{"The Future of Computers"}
\item Lecture Series on Automatic Programming, \institution{U.S. Navy Programming Schools} \dates{1952--1960}
\item Distinguished Speaker, \institution{National Bureau of Standards COBOL Conference} \dates{1960}
\end{itemize}

\section{Seminars \& Colloquia}
\begin{itemize}
\item Seminar on Compiler Construction, \institution{Harvard Computation Laboratory} \dates{1946}
\item Guest Colloquium on High-Level Languages, \institution{University of Pennsylvania} \dates{1961}
\item Seminar on Technology and Education, \institution{MIT Lincoln Laboratory} \dates{1979}
\end{itemize}

\section{Other}
\begin{itemize}
\item Namesake of the \href{https://ghc.anitab.org/}{Grace Hopper Celebration of Women in Computing}, the world's largest gathering of women technologists.
\item Namesake of the USS \textit{Hopper} (DDG-70), a guided-missile destroyer in the U.S. Navy.
\item Inducted into the National Women’s Hall of Fame \dates{1994}
\item Presidential Medal of Freedom (posthumously awarded) \institution{President Barack Obama} \dates{2016}
\end{itemize}


\begin{internalonly}
\chapter{University, College, and Department Service}

\section{University Service and Committee Work}

\begin{itemize}
\item Faculty Advisor for Mathematics Club \dates{Vassar College, 1935--1943} \\
  Organized student engagement in applied mathematics and academic competitions
\item Curriculum Committee on Wartime Education \dates{Vassar College, 1942--1943} \\
  Helped adapt curriculum to support wartime training in mathematics and computation
\end{itemize}

\section{School and Department Service and Committee Work}

\begin{itemize}
\item Chair, Mathematics Department Faculty Seminar Series \dates{Vassar College, 1938--1942}
\item Organizer, Student Colloquia on Logic and Symbolic Methods \dates{1937--1941}
\end{itemize}

\section{Other}

\begin{itemize}
\item Mentor and trainer for early women programmers in the U.S. Navy \dates{1944--1960}
\item Organizer and instructor, Naval training programs on compiler theory and COBOL \dates{1959--1970}
\end{itemize}


\chapter{Statement of Other Service Contributions}

\lipsum[1-2]  


\chapter{Student Advising}

\newcommand{\advWork}[1]{
  \begin{compactitem}[$\hookrightarrow$]
  \item #1
  \end{compactitem}
}
\newcommand{\advNote}[1]{\tabto{6.1cm}[#1]}

\section{PhD Students}

\begin{itemize}

\item Mary Van Rensselaer Buell (PhD, Navy Programming School)
  \advNote{informal doctoral-level mentorship during UNIVAC project}
  \dates{1950--1954}\\
  \hphantom{Mary Van Rensselaer Buell}
  \advWork{worked on compiler optimizations and symbolic instruction translation}
  
\end{itemize}

\section{Master's Students}

\begin{itemize}

\item Frances Holberton (MS, ENIAC Project)
  \advNote{mentored during Navy programming efforts}
  \dates{1946--1952} \\
  \hphantom{Frances Holberton}
  \advWork{developed key sorting and editing routines for COBOL}

\item Jean Bartik (MS, Moore School of Engineering)
  \advNote{early collaborator and mentee}
  \dates{1945--1950} \\
  \hphantom{Jean Bartik}
  \advWork{worked on ENIAC programming and early debugging methods}

\end{itemize}

\section{Undergraduate Students}

\begin{itemize}

\item Eleanor Kolchin (BA, Vassar College)
  \advNote{independent study in numerical methods}
  \dates{1940--1943}

\item Ruth Teitelbaum (BA, Hunter College)
  \advNote{trained under Hopper at Harvard and later the Navy}
  \dates{1945--1949}

\end{itemize}

\section{MS or PhD Thesis Committee Service}

\begin{itemize}
\item Historical mentorship role only — no formal thesis committee records available. Hopper's work predated widespread university committee structures for computing disciplines.
\end{itemize}

\end{internalonly}



\end{document}

%%%%%%%%%%%%%%%%%%%%%%%%%% End CV Document %%%%%%%%%%%%%%%%%%%%%%%%%%%%%

%----------------------------------------------------------------------%
% The following is copyright and licensing information for
% redistribution of this LaTeX source code; it also includes a liability
% statement. If this source code is not being redistributed to others,
% it may be omitted. It has no effect on the function of the above code.
%----------------------------------------------------------------------%
% Copyright (c) 2007, 2008, 2009, 2010, 2011 by Theodore P. Pavlic
%
% Unless otherwise expressly stated, this work is licensed under the
% Creative Commons Attribution-Noncommercial 3.0 United States License. To
% view a copy of this license, visit
% http://creativecommons.org/licenses/by-nc/3.0/us/ or send a letter to
% Creative Commons, 171 Second Street, Suite 300, San Francisco,
% California, 94105, USA.
%
% THE SOFTWARE IS PROVIDED "AS IS", WITHOUT WARRANTY OF ANY KIND, EXPRESS
% OR IMPLIED, INCLUDING BUT NOT LIMITED TO THE WARRANTIES OF
% MERCHANTABILITY, FITNESS FOR A PARTICULAR PURPOSE AND NONINFRINGEMENT.
% IN NO EVENT SHALL THE AUTHORS OR COPYRIGHT HOLDERS BE LIABLE FOR ANY
% CLAIM, DAMAGES OR OTHER LIABILITY, WHETHER IN AN ACTION OF CONTRACT,
% TORT OR OTHERWISE, ARISING FROM, OUT OF OR IN CONNECTION WITH THE
% SOFTWARE OR THE USE OR OTHER DEALINGS IN THE SOFTWARE.
%----------------------------------------------------------------------%
